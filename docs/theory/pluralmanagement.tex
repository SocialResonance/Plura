\documentclass{article}

% Packages
\usepackage{amsmath} % For mathematical expressions
\usepackage{amsthm} % For theorems and proofs
\usepackage{graphicx} % For figures
\usepackage{booktabs} % For professional tables
\usepackage{enumitem} % For customized lists
\usepackage{hyperref} % For hyperlinks (optional)
\usepackage[utf8]{inputenc} % For special characters

% Theorem environment setup
\newtheorem{theorem}{Theorem}[section]

% Document begins
\begin{document}

% Title and Authors
\title{Plural Management}
\author{
    Tobin South$^{1,2,\ast}$ \and
    Leon Erichsen$^2$ \and
    Shrey Jain$^2$ \and
    Petar Maymounkov$^4$ \and
    Scott Moore$^{3,5}$ \and
    E. Glen Weyl$^{2,4}$
}
\date{}

\maketitle

% Affiliations and corresponding author footnote
\begin{flushleft}
\small
$^1$ MIT \\
$^2$ Microsoft Research \\
$^3$ Gitcoin \\
$^4$ Gov4Git \\
$^5$ Metagov \\
$^\ast$ Corresponding author: \href{mailto:tsouth@mit.edu}{tsouth@mit.edu}
\end{flushleft}

% Abstract
\begin{abstract}
We introduce Plural Management, a model for partially replacing hierarchical organizational authority with plural mechanisms allowing networked authority. Participants earn influence by anticipating and fulfilling organizational priorities and harness this influence to set priorities and validate contributions, fostering a dynamic, merit-based power structure. This approach, which we illustrate with the example of open-source software development, emphasizes valued contributions and diligence without requiring hierarchical choke points, thereby enhancing participation and allowing adaptive collective intelligence.
\end{abstract}

% Protocol Overview (unnumbered section)
\section*{Plural Management Protocol Overview}
Members of an organization can dynamically earn management credits through work. An issue board allows members to prioritize issues using quadratic funding with credits. These credits will pay members who address these issues with contributions. The payout only happens after a quadratic vote, held by other members who spend their management credits to exercise authority. Organization administrators can choose to allow members to earn management credits for correctly predicting vote outcomes, rewarding individuals for due diligence on contributions, and for anticipating the preferences of existing managers. This protocol allows dynamic management control through mechanisms that scale from small collections of members to larger organizations without simplistic hierarchies.

% Section 1: Introduction
\section{Introduction}
The standard dichotomy between the rigidity of hierarchical organizations and the fluidity of flat configurations is a basic challenge for organization design. Traditional hierarchies with clear command structures are still the norm but are often seen as stifling the dynamic capabilities organizations need to thrive in today's complex landscape. Conversely, flat structures, while inclusive and dynamic, often struggle to maintain coherent direction, momentum, and accountability, often falling into the ``tyranny of structurelessness'' \cite{ostrom2011private}.

The classic alternative to this dichotomy is the use of markets \cite{hamel2020humanocracy,coase1995nature}. Yet, a critical role of firms is to create internal partially public goods and take advantage of increasing returns, which markets generally do not efficiently supply \cite{samuelson1995diagrammatic}. Thus \cite{groves1973incentives} and \cite{groves1979incentives} argued for using public goods mechanisms to organize production inside firms in lieu of hierarchies and markets. Yet, these mechanisms have typically been seen as cumbersome and impractical.

Recently, however, variants on these public good mechanisms, especially Quadratic Voting \cite{lalley2016quadratic} and Quadratic Funding \cite{buterin2019flexible}, have been increasingly broadly and successfully applied\footnote{See examples such as Gitcoin or giv for instance.}. This paper seeks to harness these advances to return to Groves's agenda and outline a framework we call ``Plural Management'' that combines these with other successful mechanisms to mimic many features of organizational authority and collaboration without requiring simplistic hierarchy.

Traditional hierarchical management systems, the backbone of modern corporate and organizational structure, are predicated on power dynamics that traditionally follow a top-down approach \cite{drucker1974political}. Employees within such systems often climb the ladder by demonstrating value through hard work and alignment of their actions with a culture articulated by those in authority, a practice sometimes pejoratively caricatured as `sucking up.' Detractors note that such practices can suppress creativity, reduce employee engagement, create bottlenecks in decision-making, and often result in the underutilization of talent at lower levels of the organization.

At the other end of the spectrum, the absence of structured management has its pitfalls, such as the `tyranny of structurelessness,' where the lack of clear roles and responsibilities can lead to chaos, inefficiency, and the emergence of informal and often unaccountable power structures \cite{friedman2007social}. Striking a balance between overly rigid hierarchies and a complete lack of structure has been a complex endeavor. Several innovative management approaches have been proposed and widely implemented, such as flat organizations that minimize hierarchical levels \cite{laloux2014reinventing}, holacracy which distributes decision-making through overlapping teams \cite{robertson2015holacracy}, and sociocracy which emphasizes consensus in governance \cite{buck2012creative}. Each of these models seeks to address the limitations of traditional hierarchy by promoting a more egalitarian and adaptive approach \cite{rothschild1986cooperative}. None, however, has the mechanistic clarity of either markets or hierarchies, arguably undermining their capacity to avoid the challenges Freeman highlights. We aim to harness advances in plural mechanism design to fill this lacuna.

In the proposed model, management credits serve as a dynamic ledger of contribution and influence. Internal, non-financial credits are initially assigned based on role or past contributions and are subsequently earned through direct contributions and triage. The expenditure of these credits in setting priorities and approving contributions is governed by a quadratic cost function, steering towards optimal public goods outcomes by avoiding excessive dominance of those with greater authority. Prediction markets are used to encourage those with limited authority to act as ``analysts'', helping authorities triage contributions, and dynamic evolution of priorities on unaddressed tasks act as a sort of dynamic auction-like bounty system to ensure tasks are addressed in a timely and well-prioritized manner.

We thus aim to combine the flexibility and dynamism of flat structures, the clear and transparent incentives of markets, and the collective orientation and strategic direction provided by traditional management. While for concreteness we focus on a fully specified design that will be deployed in several near-term applications that we discuss, we fully expect significant further improvement on these mechanisms as we discuss in the conclusion and thus aim to suggest as much a general structure for combining mechanisms to achieve this synthesis as the specific design we use to illustrate this structure.

The rest of this paper is organized as follows: Section 2 presents the Plural Management Protocol, detailing the high-level description of the system, roles within the ecosystem, and the processes of earning and spending management credits. It goes on to discuss the practical application of the Plural Management system in the context of open-source software development, illustrating how it can help address long-standing problems of management in open-source projects as they scale. Section 3 elaborates on this to provide a more detailed technical version of the protocol suitable for implementation. In Section 4, we provide a detailed analysis of the protocol properties, examining the voting and prediction behaviors and the optimal parameter choices within the system. Lastly, Section 5 examines some implications of such a protocol and highlights the open questions and future work it presents.

% Section 2: Plural Management Protocol
\section{Plural Management Protocol}
There are three roles in the ecosystem: workers, who make direct contributions to an organization; managers, who determine what work is important and whether a piece of work is of acceptable quality; and administrators, who can determine system properties to sway behavior. Importantly, an individual can act in multiple of these roles at any given time and in relation to multiple other individuals; no role is fixed and members of an organization are encouraged to act in a diversity of roles with respect to a diversity of other individuals.

Instead of a set of hierarchical roles assigned to an individual, each person has a set of management credits for this organization. These credits allow an individual to exercise authority in decision-making and be recognized for contributions made. We walk through each step where credits are gained and spent. These credits are specific only to a particular organization, project, or community and have no value outside of it; in this sense they are similar to a ``community'' or ``artificial'' currency \cite{blanc2018making}. As we discuss later, these credits cannot be traded externally; they serve only to control the flow of dynamic management potential.

Consider an organization that has an issue board where all major tasks or initiatives to be completed are listed (similar to open-source issue trackers as in subsection 2.1). Individuals with management credits can set priorities for an organization by assigning a priority value for an issue using credits. The priority on an issue is not simply the sum of the credits assigned to it, but instead may be matched by a matching pool (provided by individuals in their role as administrators) consistent with practical applications of Quadratic Funding, as we will discuss further below.

An individual may perform the role of a worker and provide a solution to an issue in the form of a contribution. If this contribution is accepted, the worker will receive credits proportional to the total number of credits assigned in the priority setting. In essence, from a worker's perspective, the `bounty' attached to an issue may go up over time similar to a reverse Dutch auction, until a sufficient reward is offered to compensate the cost to the worker of addressing the issue, though there is not a necessary guarantee that the reward will increase over time.

Once a contribution has been made, it goes to a contribution vote. In this vote individuals can expend management credits to vote if the contribution should be approved or not. If the vote passes, the worker is rewarded; if not, the issue returns to the board (wherein managers can increase the priority to provide a higher bounty). This vote is done quadratically, this ensures a balanced impact between individuals with varying credit amounts.

In addition to casting a vote, each individual can choose to `bet' the number of credits they used to vote that their prediction of acceptance or rejection will succeed. If correct, this bet will pay out double the credits used to vote. This vote prediction allows individuals to be rewarded for correctly anticipating the desires of the full community. We introduce a prediction subsidy parameter that can be set by administrators for each contribution vote that reduces the cost of voting and increases the reward from betting. By default, voting and then betting on this is strictly unprofitable. However, in many cases, administrators may wish to increase the subsidy to allow opportunities for individuals who can anticipate community needs to gain authority. For example, providing a subsidy can incent individuals with fewer management credits to participate in votes that would otherwise be costly, which means that a larger crowd of individuals is performing due diligence on contributions. If many contributions are being made in a large organization, this is akin to rewarding individuals for administrative processing that ``triages'' contributions and thus surfaces important divisive votes to managers with more authority.

Put together, these two systems of quadratic agenda setting and hybrid voting-prediction can create a dynamic system of management, where contributions are rewarded in proportion to their public good demand when the broader organization collectively approves of them and individuals who have developed a robust understanding or model of the community preferences are rewarded and empowered for supporting administrative processes.

\subsection{An Application to Open-Source}
Although the plural management protocol can be applied across a wide range of organizations and communities, it has particular relevance to the world of open-source software and other spaces where peer production is common \cite{benkler2017peer}. Far from being a niche industry, git-based open-source powers over 93\% of all modern software applications \cite{daigle2023octoverse}, and already operates via community models of governance, where contributions in the form of code are assessed for quality and relevance before being merged into existing work. Despite these important contributions, open-source communities are well-known for their governance and management challenges, documented most famously by \cite{eghbal2020working} and including the following:

\begin{enumerate}
    \item While the contributions of open-source contributors are recorded, recognition is hard to track/trace because contributions are not clearly valued in relation to higher-level objectives. This reduces motivation and sustainability.
    \item While contributions to open source projects are generally open and participatory, management (often called ``maintenance'') of them usually falls in the hands of a ``benevolent dictator for life'', contradicting the underlying democratic values and leading those who dissent to ``fork'' projects, fragmenting efforts.
    \item Worse, the inability to leverage distributed participation to assist in management makes projects large burdens on maintainers, who begin projects with high motivation but are forced to maintain their quality for years after, forcing them to triage increasing volumes of contributions of dubious quality with little community support.
    \item Especially as they grow and are more broadly used, potential directions for improvement of a project grow exponentially and there is typically little clarity on what improvements are most needed by users, leading to projects that have too many features and insufficient usability.
\end{enumerate}

By offering greater clarity and empowerment to contributors, plural management can help founders to slowly transition management authority to those who prove their merit by contributing code, diligence, or support in a way that is measurably valuable to the community in question. Since the model is lightweight, iterative, and self-directed, it is well suited for commonly used agile environments and tools of the kinds often used by open-source communities.

Consider the most popular open-source hosting platform GitHub. For any given project there is a repository of code, set up by an administrator or maintainer, within which any contributor is listed as a member. Attached to this repository is an Issues section (extremely similar to our described issue board, just without any explicit priorities assigned numerically). Anyone on GitHub can create a contribution on this board in the form of a `pull request' (shorted to PR) that aims to address one or more outstanding issues. After the pull request is discussed in a comment section, the community can decide whether to accept or reject it, and in turn, maintainers can add or `merge' the contribution into the repository. Using Plural Management, with minimal changes to workflows, any maintainer or administrator could set priority tags on GitHub, associate a price in credits, and in turn drive more contributions to their repository as a first step in eventually improving their bus factor from the low average of two \cite{metabase2022bus}.

It's worth noting for the general case that, while contributions are typically code, anything could be made a PR. For example, if someone were to be appointed to the social media manager for a project, an issue stating the need for a social media manager could be made, and when someone is to be appointed, a simple PR adding the name of the person to the community notes could be made by the new social media manager. If the community votes to approve this new role, the social media manager will now be rewarded with additional management credits reflecting their new role.

\subsection{A Succinct Example of Use}
A tangible example of the use of plural management beyond the usual open-source context is the plurality book, an open, git-based experiment around collective authorship. Initiated by E. Glen Weyl and Audrey Tang, 50 members around the world have contributed to the book, \textit{Plurality: The Future of Collaborative Technology and Democracy} without any expectation of reward. Using the plural management protocol, this project seeks to transition ownership over future improvements to the book incrementally, including updates to content, translations, and further links to relevant materials. Over time, those who have contributed most meaningfully will therefore help guide not just the book but the field of research itself.

Consider an undergraduate student of political economy at a lesser-known university. Seeing a typo, she opens up an issue and submits a PR. This action does not net many credits during voting, but the small number she is given allows her to begin participating in priority setting. Motivated, she continues searching for opportunities to contribute and recognizes that an outstanding issue around additional content for a chapter could benefit from her thesis work. She submits a PR on the board to add a number of key references that get cited in the book and is rewarded with significant credits.

Given the existing challenges of inclusion within the post-secondary context, without the permissionless and community-judged power structure afforded by plural management such a student may never have had the opportunity to participate in such work \cite{gvozdanovic2018implicit}.

% Table 1
\begin{table}[h]
    \centering
    \caption{Summary of Protocol Mechanisms}
    \begin{tabular}{ll}
        \toprule
        Term & Description \\
        \midrule
        Management Credits & Used to exercise decision-making authority \\
        Issue Board & Lists tasks, uses credits for setting priorities \\
        Contribution & A solution to an existing issue \\
        Contribution Vote & A collective vote using credits to approve a contribution \\
        Vote Prediction & Bet on outcome of votes to earn credits \\
        Prediction Subsidy & Adjusts the cost of voting and reward for correct predictions \\
        \bottomrule
    \end{tabular}
\end{table}

% Figure 1 (placeholder)
\begin{figure}[h]
    \centering
    % \includegraphics[width=0.8\textwidth]{figure1} % Uncomment and replace with actual file
    \caption{Key components in the workflow of plural management. Any member can be a contributor or manager, where credits are spent to exercise authority and earned from contributions and correct vote predictions.}
\end{figure}

% Section 3: Model Details
\section{Model Details}
The plural management protocol describes two key activities: a prioritization subsystem and an approval subsystem. These subsystems exist jointly and constitute a broader organizational structure, where individuals earn management credits that can be used to perform actions. These credits can be initially distributed when an organization is established and are naturally distributed to new members as they participate (in effect lessening the control of founders over time). The credit can be stored in any simple ledger that can be amended over time when interacting with the protocol.

There are many additional considerations surrounding the sharing, control, and visibility of these management credits. For example, should an organization dynamically run on management credits to make the score of every member public (in essence creating a ranking of implicit authority)? Should individuals be able to directly send management credits to another member (this would ease the setup challenges for new members and allow old members to gracefully leave, but could also reduce meritocracy and result in off-the-books gambling or scheming)? We return to these open questions in our conclusion.

\subsection{Prioritization}
\textit{At a glance}

All issues are listed on a board where members can spend management credits to set issue priorities through quadratic funding (e.g., each member can spend $P_i$ credits and the total priority will be $\left( \sum_i \sqrt{P_i} \right)^2$). Large credit holders can add to a matching fund. When a contribution is made to address an issue, priority credits, and matching credits are frozen and will be allocated to the contributors if the contribution vote passes.

The first subsystem to consider is the priority-setting step via the issue board. Every major task or strategic challenge should be assigned an issue on the board, similar to how most open-source projects operate in GitHub.

Each member can spend a portion of their management credits on priority setting. This is done dynamically and members can add or withdraw credits from each issue at any time. For each member who sets a priority to an issue by assigning $P_i$ credits, we sum over the square roots of their priority and take the total square to find the total issue priority. Hence, the quadratic priority for issue $j$ is 
\[ QP_j = \left( \sum_i \sqrt{P_i} \right)^2. \]
This is exactly akin to quadratic funding, from which we further draw on the idea of a matching fund. A matching fund is generated from credits used in voting or further increased by a large management credit holder (such as an early founding member) who may choose to assign funds as a matching pool to distribute to new contributors as incentives to join.

In reality, a matching fund may not always have sufficient credits to fully subsidize the quadratic priority. To address this, the total contribution payout $(CP)$ is adjusted proportionally to the matching fund\footnote{If for each issue $j$ at time $t$ we have $QP_j$, the need for a subsidy from the matching fund is the difference between $QP_j$ and the credit put in $\operatorname{Cap}_j - \sum_i P_i^j$. Let this difference be $M_j(t) := QP_j(t) - \sum_i P_i^j$. All matching fund subsidies are then down-weighted of the priority credits and the corresponding quadratic matching $CP_j = \sum_i P_i^j + \bar{M}_j$ which is just a scaled proportion of $QP_j$.}.

When a contribution is made to address an issue, the payout should be frozen for that issue\footnote{For analysis and technical purposes, all the credits used for issue setting and the matching fund on this issue should be effectively put in escrow, and ignored from further dynamic setting. If the contribution is not accepted, the issue returns to the board with the same priority credits set (although members may then change this) and the matching funds $\bar{M}_j$ are simply added back into the matching pool total and matching is renormalized.}. The contribution then goes to a vote as below. If the vote fails, the issue simply comes back to the board for other contributions to be suggested.

Only contributions addressing existing issues are rewarded. To receive a reward for an unsolicited contribution, a contributor would need first to submit an issue and persuade the community it is worth addressing; given the quadratic nature of the matching, an individual adding an issue and contributing to it themselves can never be rewarded more than they contribute to the issue in credits. Thus an individual must persuade others of the value of their contribution in order to receive a net reward.

\subsection{Approval}
\textit{At a glance}

A contribution vote is held where any member can cast a vote of $v$ at cost $v^2$ either for or against. This is a standard single-issue quadratic vote. Organization administrators can choose to set aside funds from the issue bounty to incent members to predict a contribution's likelihood of success, rewarding them for correct predictions. This helps incent small credit holders to understand broader organization needs and perform due diligence on contributions. Predictors will get a $2v$ payout and administrators can reduce the cost of voting relative to predicting by setting $K$.

Once a contribution is made, it goes to a vote. Any member with management credits can vote, and in line with the quadratic voting approach, it will cost $v^2$ credits for a vote of strength $v$. Any individual can vote for or against, where votes against can be treated at a negative value of $v$ for the sake of determining the outcome. As with any quadratic vote, there should be an appropriate time to allow for the vote to occur and the verdict after this time is simply the sum of the votes. In the simple case, all of the funds put into the issue during priority setting go to the contributor. All the credits used by members during voting flow directly into the general matching fund for priority setting \footnote{The making voting costs of the effect of the matching fund, this management credit ecosystem becomes entirely self-contained. This is useful, as credits are never lost, and over time they generally flow to those who are contributing and managing. If more credits are needed in a system administrators could change their parameter choices or add more into the system in an inflationary act.}.

This act of simply voting is costly, meaning that members can spend earned trust to exert authority and influence the direction of the organization or project. As a result, an early member of the community with few credits would find it proportionally quite costly to sway a vote relative to their means. This further means that members with few management credits have no incentive to do due diligence on contributions (often a significant workload) to determine their fit for the project.

In order to incent lower authority members to participate in the vote and provide a signal of quality for the contribution, organization administrators can reward an additional prediction step by voters. Administrators will choose a parameter $K$ that reduces the cost of voting relative to prediction. The cost of voting will be $K v^2$ if no prediction of a correct vote is made, or $K v^2 + v$ if a wager of $v$ is made alongside the vote. The payouts for these correct predictions will come from the contribution payout and can be seen as a processing fee to incent analysis of the contribution.

Predictors can choose to make no prediction without cost or wager exactly $v$ additional credits that the outcome they voted for will succeed \footnote{Individuals can only be in the direction of their vote and in an amount capped by their vote, as this incents approximately truthful predictions from the logic of quadratic scoring rules \cite{selten1998axiomatic}.} for a payout of $2v$ \footnote{Given no hedging and a quadratic vote, we will see in Theorem 1 that given the ability to wager any number between 0 and $v$, it is always optimal to wager $v$ when you believe the probability of success is greater than a half.}. For large values of $K$ (e.g., one or greater), voting would only be profitable for exceptionally small numbers of credits due to the quadratic cost of voting (although one can still minimize their losses by wagering $v$ credits when the likelihood of passing is believed to be greater than $\frac{1}{3}$). When $K=0$, there is no cost to voting and one should (if one is risk-neutral) wager the maximum credits if they believe their vote will pass. This setting of $K=0$ should be avoided and indeed Theorem 3 will show that $K$ can and should be set high enough not to reduce the contribution payout more than expected. In general, administrators can learn over time what choices of $K$ are needed to incent different mixtures of behavior.

An important aspect of the mechanism is that this prediction reward is only positive for small votes. Due to the squared cost of voting, large votes with high impact will not ever be profitable for reasonable non-zero values of $K$. This is an important property such that existing authority figures with large sums of management credits are not rewarded for understanding the community preferences and rewards go to those who are seeking to increase their influence from modest means.

% Table 2
\begin{table}[h]
    \centering
    \caption{The processes where a member of the community can receive (sources) or spend (sinks) management credits.}
    \begin{tabular}{ll}
        \toprule
         & Contribution Acceptance Vote & Issue Priority Setting \\
        \midrule
        Sinks & Voting on outcome & Voting on priorities \\
        Sources & Correct verdict prediction & Accepted contribution to solve issue \\
        \bottomrule
    \end{tabular}
\end{table}

% Section 4: Analysis of Protocol Properties
\section{Analysis of Protocol Properties}
Here we perform an analysis of the protocol voting behavior to show its properties and derive parameter choices that achieve various goals or guarantees. We use the construction and definitions above and introduce the assumption that each individual always acts to maximize a personal utility which is a linear combination of outcomes going according to one's preferences and earning more management credits (which in turn affords more power to sway future outcomes). This analysis is analogous to the usual quasilinear utility set-up applied in mechanism design and is appropriate under conditions discussed, for example, by \cite{buterin2019flexible}. While \cite{gorokh2021from} shows this extension may approximately apply in long-running private goods economies, it is much less clear it is relevant in primarily public goods economies like those we are studying. Nonetheless, it is a standard starting point for analyzing behavior in these contexts. We do not explicitly model the behavior of ``administrators'' who supply matching funds to support others' preferences, but such behavior can be understood in terms of acting in the common interests of the community for altruistic, ideological or un-modeled pecuniary reasons (e.g., the administrator may hold equity in the productive output of the group)\footnote{Remembering that these management credits have no value outside of the organization, project, or community in which they are based, distributing management control is beneficial to founding or long-time members if it increases the likelihood of the group mission continuing, or more simply, increases the number of participants in management, creating less work for senior members.}.

We work through a series of theorems and proofs to support the construction above and show the effect of specific value choices. First, we focus on the voting prediction step.

\begin{theorem}[Individuals will always wager either 0 or $v$ credits]
For a given contribution vote, a rational individual maximizing credits with $v$ votes cast at cost $v^2$ will wager 0 credits on their desired outcome succeeding if the likelihood is less than half or $v$ additional credits for likelihoods greater than half.
\end{theorem}

\begin{proof}
Conditional on an individual already having cast a vote of $v$, the return on the vote-prediction round will be $2w$ credits. Since the vote has a binary outcome that will return either $2w$ or 0, the expected return for an individual is $2w\rho$ for a probability of success $\rho$. Hence, subtracting the costs of voting and wagering, the total expected profit is 
\[ 2w\rho - K v^2 - w = 2\left(\rho - \frac{1}{2}\right)w - K v^2. \]

Conditioned on $v$, the expected profit grows linearly positive with $w$ so long as the expected probability of your preference occurring is above $\frac{1}{2}$. Given that $K v^2$ has already been paid as a fixed cost here, the individual should maximize $w$ in any case where $\rho > \frac{1}{2}$.
\end{proof}

This gives the intuitive result that a pure profit maximizing predictor should wager the max value possible $w=v$ if the expected probability of winning is greater than 50\%, and wager nothing otherwise.

Regardless of this strategy (as we see in Figure 2), for $K=1$ there is never any positive profit from voting, and indeed the returns from betting become extremely small compared to the quadratic cost when a voter cares strongly about an outcome. In general, a voter can choose to wager with their vote if they believe it will succeed in minimizing their costs. For small values of $K$, those placing small votes can make a profit in management credits.

Indeed for a purely profit-maximizing individual, there is an optimal choice of $v$.

\begin{theorem}[Optimal $v$]
For a given choice of $K$, an individual who is exclusively profit maximizing should cast a vote $v = \frac{\rho - 1}{K}$ for $\rho$ greater than half and zero otherwise.
\end{theorem}

\begin{proof}
An individual who is exclusively profit maximizing should vote to maximize the profit function $2\rho v - K v^2 - v$. This is achieved at $v = \frac{\rho - 1}{K}$ for $\rho$ greater than half and zero otherwise.
\end{proof}

Using this result we can ensure that the contribution payout isn't unduly drained in subsiding prediction.

\begin{theorem}[Don't drain the contribution payout]
A conservative choice of $K$ can be made to ensure that no more than an $\alpha$ proportion of the contribution payout is ever spent rewarding prediction.
\end{theorem}

\begin{proof}
For $N$ total purely credit maximizing individuals holding credits who can vote and a choice of $K$, if each individual knew exactly how to predict such that they only wagered if they were to win, then each individual who won would vote and wager with $v = \frac{1}{2 K}$ knowing they would win, giving them a payout of $\frac{1}{K}$ and a profit of $\frac{1}{2 K} - K \frac{1}{(2 K)^2} - \frac{1}{2 K} = \frac{1}{4 K}$. As a result, the full drain on the contribution payout from the priority setting would be $\frac{N}{K}$.

Given one does not want to drain more than an $\alpha$ proportion of the contribution payout $CP_j$, $K$ should be set such that $\alpha CP_j > \frac{N}{K}$. This maximum payout would only occur if all members holding credits voted in the same direction.
\end{proof}

It is possible to exceed this total prediction payout in the circumstance where a large credit holder who already knew a vote would pass (given every member is voting in favor) voted and predicted at a high cost of credits in order to further drain the contribution payout. Given the circumstances needed to achieve this, it would likely be done with the intent to reduce the reward to the contributor and the voter's own cost. Indeed, the vote would have to expend an extremely large number of voting credits due to the quadratic cost, making this unlikely.

What we see from this analysis is that for a reasonable value of $K$ above 0, small votes will be rewarded a small amount, and large votes will almost always be costly. This is an important effect of the design, creating a quadratic penalty that rewards many individuals participating, and further rewards small voters for their involvement in decision-making, while minimizing a pure credit-seeking incentive for large credit-holding existing members of the community.

% Figure 2 (placeholder)
\begin{figure}[h]
    \centering
    % \includegraphics[width=0.8\textwidth]{figure2} % Uncomment and replace with actual file
    \caption{Top Left: the optimal value of $b$ for maximizing profit based on different values of $\rho$ for $K=v=1$. $b$ should be set to $v$ if $\rho > \frac{1}{2}$ and 0 otherwise. For $K=1$ the profit is strictly negative. Top Right: for smaller values of $K$, larger bets can return positive values for correct predictions. $K$ can be chosen to incentivize or reduce pure profit-seeking behavior. Bottom: The maximum profit for member voting casting a vote of $v$ and making a corresponding prediction.}
\end{figure}

\subsection{Mixed Utility Analysis}
While a credit return maximizing analysis is helpful in understanding incentives and behavior, it is certainly true that individuals are not solely motivated by gaining credits. Indeed, the very reason to gain credits is to enact preferences or beliefs on future votes.

Instead, we can define a utility $(U)$ that is constructed from your management credits profit for any given vote with the addition of one's desire for a preferred outcome $(\gamma)$ with a binary indicator of achieving that outcome $(\mathcal{A})$:
\[ U = \gamma \mathcal{A} + 2 w \rho - K v^2 - w. \]

From this, we see that when maximizing the bet over mixed utility, we still find the same optimal points of betting at either 0 or $v$ (this follows naturally from $\frac{d}{dw} \gamma \mathcal{A} = 0$, or restated, that your bet has no impact on the outcome of a vote).

Now consider an individual's pivotality, $\phi := \frac{d \mathcal{A}}{dv}$, as the capacity for an individual to change the outcome of vote $\mathcal{A}$. For notational convenience, we introduce $W := 1$ if $\rho > \frac{1}{2}$ else 0, to simplify the process of taking derivatives over $w$ which will be either $v$ or 0.

\begin{theorem}[Optimal $v$ under mixed utility]
For a given choice of $K$, an individual should vote $\frac{\gamma \phi}{2 K}$ when their outcome is unlikely $\left( \rho < \frac{1}{2} \right)$ and add $\frac{\rho - \frac{1}{2}}{K}$ for added rewards when the desired outcome is deemed likely.
\end{theorem}

\begin{proof}
We maximize the mixed utility with respect to $v$:
\[
\begin{aligned}
\frac{d U}{dv} &= \gamma \phi + 2 \rho W - 2 K v - W \\
0 &= \gamma \phi + (2 \rho - 1) W - 2 K v \\
v &= \frac{\gamma \phi + (2 \rho - 1) W}{2 K}
\end{aligned}
\]

This gives a similar result to Theorem 2, but now with an added quantity for $v$ when $\gamma > 0$. In essence, if a member has no preference over the outcome $(\gamma = 0)$ they should vote according to the previous credit-return-maximizing approach given $K$. If there is a preference, they should vote according to that level of desire, and the likelihood by which their vote for a choice of $v$ will change the outcome of the vote.
\end{proof}

\subsection{A Detailed Illustrative Example}
Since the numbers and analysis can seem abstract, we walk through a small constructed example in the interest of clarity.

Suppose there is a community of one founder with 2000 credits and eight members with 1000 credits earned through contributions. Here there is a total pool of 10,000 credits. The founder creates a matching fund of 1000 credits to support newcomers. On the issue board, there are ten existing issues on the project. One such issue has been assigned a priority of 5 by each of the 8 members besides the founder (costing a total of 25 credits each). The total pool of credits for this issue is $\operatorname{Cap}_j(t) = 5^2 \times 8 = 200$, while the quadratic priority of the issue is $QF_j(t) = (5 \times 8)^2 = 1,600$. Since there are not enough credits in the matching pool to provide for this, all the contribution payouts will be adjusted down. If we imagine all the issue priorities are set evenly, then the reduction would come out to $k=0.1$. As a result, the contribution payout is $CP_j(t) = 200 + 100 = 300$. All of these calculations are done automatically and the members simply see the total contribution bounties for each issue.

A new member joins the community by providing a contribution to this issue. Seeing a high bounty helped motivate this choice of issue over others in the community, but the choice to join the community is only out of interest. These management credits are only useful within this project.

The contribution goes to a vote, but no members want to vote as they have already expended many credits, and examining the contribution is laborious. To incentivize this, the administrator (here the founder) can set $K=0.1$ to reward prediction. Each member examines the contribution, with half casting a vote of 5 against and the other half for. Since $K=0.1$, this cost each member $0.1 \times 5^2 + 5 = 7.5$ to predict and cast their bet (since all members were very confident in themselves). The founder comes in and places an unnecessarily large vote of 12, costing $0.1 \times 12^2 = 14.4$. The vote passes and each member who voted correctly gets a payout of $2 \times 5 = 10$ from the bounty (draining the contribution payout by only 40 credits, essentially a processing fee for participation in voting and performing due diligence).

With a passing contribution, the 40 credits go to the correctly predicting members and $300 - 40 = 260$ credits go to the contributor, who can now use his credits in future votes. All the credits spent on voting $(14.4 + 7.5 \times 8 = 74.4)$ go back into the matching fund to incentivize future contributions.

% Section 5: Open Questions
\section{Open Questions}
Plural Management, while versatile in theory, encounters practical challenges in diverse organizational scenarios. Its adaptability to both modern, open-source environments and traditional hierarchical structures raises questions about its real-world efficacy and implementation strategies. This section probes into these nuances, inviting deeper exploration and collaborative research to navigate the complexities of applying Plural Management in varying contexts.

\begin{enumerate}
    \item In refining Plural Management's quadratic voting, it's crucial to recognize and strategically address the potential for collusion within homogeneous socio-cultural groups in organizations, considering dimensions like location, department, role, and origin. Building on foundational research \cite{millerBeyond}, this approach advocates for a nuanced mechanism that actively discounts the disproportionate influence of these groups. Such a system would not only enhance fairness but also promote a genuinely diverse and representative decision-making process, ensuring that no single faction within the organizational tapestry exerts undue control, thus aligning more closely with the realities of complex organizational structures.
    \item Can we extend this approach to create a multilayered decision-making framework within organizations? This would involve developing independent yet interconnected systems for various organizational strata, such as departments or teams, each with its own tailored voting mechanism. Such a model could facilitate more localized and relevant decision-making, while maintaining coherence with the broader organizational objectives. This approach merits exploration for its potential to harmonize individual group dynamics with the overall organizational structure.
    \item While rational actors may be further motivated by extrinsic incentives, their overuse has the potential to create what is referred to as `motivation crowding' in which intrinsically driven contributors are discouraged from participating in a project \cite{frey2000motivation}. While management credits are not by default monetary, the decision to distribute such rewards based on them must therefore be made in the context of existing organizational cultures.
    \item The transparency of status hierarchies within organizations, as is often studied with respect to salaries, can have meaningful impacts on contributor behavior \cite{cullen2023pay}. While this can improve employee outcomes, it can also reduce internal collaboration and potentially harm long-term organizational objectives. Given its ease of implementation, plural management provides a sandbox for comparing and contrasting how public or private records of credits impact performance within a group.
    \item Currently, individuals cannot directly send management credits to another member, as a way to prevent off-book trading of management credit which could result in a market price for management authority. This helps prevent the financialization of management authority, and makes certain behaviors difficult or impossible. For example, if a founder wants to bring in a new member quickly to provide authority, they cannot directly send credits and must conduct elaborate PR rewards for this contributor that are voted on by the whole community (in some sense also preventing nepotism). Further, if a member wants to leave, they cannot transfer credits to others quickly aside from putting all credits in a matching fund. Future research into the implications of enabling direct trading is important before incorporating it.
    \item Deciding when and how to promote an employee is a mission-critical question across organizations. However, in large hierarchies employees are often promoted based on their performance in an existing role rather than their capacity to set high-level priorities leading to bad management \cite{benson2018promotions}. An evaluation of the effects plural management has on promotion results, for example by evaluating how top contributors perform in administrator roles, would be helpful.
    \item The current design of plural management is focused on a single organization, community, or project. Many large organizations are constructed as many suborganizations working together in the form of departments, units, or project teams. Future work could examine the use of plural management to create multiple partially nested versions that allow management authority to be exercised within a sub-organization, while still allowing individuals to climb the ranks of larger workplaces.
    \item Negative voting can provide a useful signal, but also has the potential to create polarization within group contexts \cite{weber2021negative}. This has been observed empirically within the context of quadratic funding rounds, such as those run by Gitcoin \cite{buterin2020gitcoin}. Running instances of plural management with and without negative voting could help further evaluate its psychological effects and impact on cooperative behavior.
    \item When the result of a prediction market can be influenced by the members participating in it, the possibility of collusion to manipulate outcomes arises \cite{ottaviani2007outcome}. An analysis of how participants in plural management vote with and without the opportunity to predict outcomes would be useful for understanding what if any limitations should be placed on rewards from this activity.
\end{enumerate}

% Section 6: Conclusion
\section{Conclusion}
Plural management is a protocol for bridging between the desired properties of rigid management hierarchies and flat decentralized organizations, allowing for the dynamic allocation of management authority based on longitudinal contributions of individuals to outcomes and management decisions. Through the adjustment of a voting-prediction discount parameter, administrators can reward new or low credit-holding members of a community for their work in performing due diligence of new contributions in line with the expected standards or higher-authority members. This management approach, which uses quadratic funding to solicit preference from a broad base of participants, creates a closed credit system with no external monetary value, that can only be exercised within the project. While this design of a management protocol raises open questions about implementation choices and net outcomes on organizations' productivity, it can be built using standard software design practices and fits naturally into the workflow of open-source projects. In total, this model of plural management could present a dynamic scalable approach to distributing authority and rewarding participation across projects of any scope, mission, or size.

% References
\begin{thebibliography}{99}
\bibitem{benkler2017peer}
Yochai Benkler. Peer production, the commons, and the future of the firm. \textit{Strategic Organization}, 15(2):264--274, 2017.

\bibitem{benson2018promotions}
Alan Benson, Danielle Li, and Kelly Shue. Promotions and the peter principle. \textit{National Bureau of Economic Research}, 2018.

\bibitem{blanc2018making}
Jérôme Blanc. Making sense of the plurality of money: a Polanyian attempt, pages 48--66. \textit{Financial history}. Routledge, 2018.

\bibitem{buck2012creative}
John A Buck and Gerard Endenburg. The creative forces of self-organization. \textit{Sociocratic Center, Rotterdam, The Netherlands, Tech. Rep}, 2012.

\bibitem{buterin2020gitcoin}
Vitalik Buterin. Gitcoin grants round 5 retrospective, Apr 2020.

\bibitem{buterin2019flexible}
Vitalik Buterin, Zoë Hitzig, and E Glen Weyl. A flexible design for funding public goods. \textit{Management Science}, 65(11):5171--5187, 2019.

\bibitem{coase1995nature}
Ronald Harry Coase. The nature of the firm. \textit{Springer}, 1995.

\bibitem{cullen2023pay}
Zoe Cullen. Is pay transparency good? \textit{National Bureau of Economic Research}, 2023.

\bibitem{daigle2023octoverse}
Kyle Daigle. Octoverse: The state of open source and rise of ai in 2023, Nov 2023.

\bibitem{drucker1974political}
Henry Matthew Drucker. The political uses of ideology. \textit{Springer}, 1974.

\bibitem{eghbal2020working}
Nadia Eghbal. Working in public: the making and maintenance of open source software. \textit{Stripe Press}, 2020.

\bibitem{frey2000motivation}
Bruno S Frey and Reto Jegen. Motivation crowding theory: A survey of empirical evidence, revised version. \textit{Working paper series/Institute for Empirical Research in Economics}, (49), 2000.

\bibitem{friedman2007social}
Milton Friedman. The social responsibility of business is to increase its profits. In \textit{Corporate ethics and corporate governance}, pages 173--178. Springer, 2007.

\bibitem{gorokh2021from}
Artur Gorokh, Siddhartha Banerjee, and Krishnamurthy Iyer. From monetary to nonmonetary mechanism design via artificial currencies. \textit{Mathematics of Operations Research}, 46(3):835--855, 2021.

\bibitem{groves1973incentives}
Theodore Groves. Incentives in teams. \textit{Econometrica: Journal of the Econometric Society}, pages 617--631, 1973.

\bibitem{groves1979incentives}
Theodore Groves and Martin Loeb. Incentives in a divisionalized firm. \textit{Management Science}, 25(3):221--230, 1979.

\bibitem{gvozdanovic2018implicit}
Jadranka Gvozdanović and Katrien Maes. Implicit bias in academia: A challenge to the meritocratic principle and to women's careers-and what to do about it. \textit{League of European Research Universities (LERU) Advice Paper No}, 23, 2018.

\bibitem{hamel2020humanocracy}
Gary Hamel and Michele Zanini. Humanocracy: Creating organizations as amazing as the people inside them. \textit{Harvard Business Press}, 2020.

\bibitem{lalley2016quadratic}
Steven P Lalley, E Glen Weyl, et al. Quadratic voting. \textit{Available at SSRN}, 2016.

\bibitem{laloux2014reinventing}
Frederic Laloux. Reinventing organizations. \textit{Nelson Parker Brussels}, 2014.

\bibitem{metabase2022bus}
Metabase. Bus factor of top github projects, Nov 2022.

\bibitem{millerBeyond}
Joel Miller, E. Glen Weyl, and Leon Erichsen. Beyond collusion resistance: Leveraging social information for plural funding and voting. \textit{SSRN}.

\bibitem{ostrom2011private}
Elinor Ostrom and Charlotte Hess. Private and common property rights. In \textit{Encyclopedia of law and economics}. Edward Elgar Publishing Limited, 2011.

\bibitem{ottaviani2007outcome}
Marco Ottaviani and Peter Norman Sørensen. Outcome manipulation in corporate prediction markets. \textit{Journal of the European Economic Association}, 5(2-3):554--563, 2007.

\bibitem{robertson2015holacracy}
Brian J Robertson. Holacracy: The new management system for a rapidly changing world. \textit{Henry Holt and Company}, 2015.

\bibitem{rothschild1986cooperative}
Joyce Rothschild and J Allen Whitt. The cooperative workplace: Potentials and dilemmas of organisational democracy and participation. \textit{CUP Archive}, 1986.

\bibitem{samuelson1995diagrammatic}
Paul A Samuelson. Diagrammatic exposition of a theory of public expenditure. \textit{Springer}, 1995.

\bibitem{selten1998axiomatic}
Reinhard Selten. Axiomatic characterization of the quadratic scoring rule. \textit{Experimental Economics}, 1:43--61, 1998.

\bibitem{weber2021negative}
Till Weber. Negative voting and party polarization: A classic tragedy. \textit{Electoral Studies}, 71(3):221--230, 2021.
\end{thebibliography}

\end{document}